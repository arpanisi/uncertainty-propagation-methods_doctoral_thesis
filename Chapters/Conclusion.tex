\chapter{Conclusion and Future Works}
\label{chap:conclusion}

In this chapter, insights gathered from previous chapters are summarized. Additionally, avenues for future research directions are also outlined. Chapter~\ref{chap:wcs} discusses the details of \textit{linearization and clustering} methods that has been used throughout the dissertation. Answers pertaining to several questions about the usage of \textit{linearization and clustering} methodology are outlined next:

\textit{How do the differences in the methods of Linearization effect the overall performance of propagation of uncertainty?} The results of clustering differ with different methods of linearization. The concept of average linearization is based on the assumption that the methodology is able to capture sufficient information for the clustering algorithm to be implemented. The concept of average Eigenvalues differ from the concept of average Jacobian in terms of periodicity of the system. If the system exhibits a limit cycle oscillation, then the ideal time window for averaging out the Eigenvalues should be the time period of the oscillation. This time window differs from problem to problem. Averaging out the Jacobian may not require such condition. However, one should notice that, while using linear coupling function, the Jacobian does not take into consideration the change in the state variables due to coupling. Hence, the detection of cluster will depend only on the value of the constant coupling strength and not on the time window. The Statistical Linearization is however a Jacobian free method and independent of the time window. As discussed earlier, it is the only method which is able to capture the effect of changing covariance, which may affect the result. Performance of both type of methods do not differ much, which establishes the fact that the Statistical Linearization is able to capture the same information of a nonlinear system as the Jacobian-based techniques. The method is also easier to implement, as it does not rely on the nominal trajectory as well. 

\textit{Which one of the seven methods is suitable?} Our work has proposed seven techniques combining two methods of graph clustering and four methods of linearization. From the ANOVA Table~\ref{anova}, it is evident that the contribution of the techniques towards the variability of the results is not high. Thus, the choice of the method entirely depends on two factors i) computational expense and ii) Reliability of the methods.

\textit{How are the performance and the reliability of the seven methods:} The linearization and clustering method together work excellent for weakly coupled oscillators. The error values for all the test cases with $e = 0$ are negligible. As pointed out earlier, the clustering technique accurately identifies the weakly coupled oscillators at each of the time instance. The linearization step involves the Jacobian computation, that captures the block-diagonal structure. Due to numerical values it may happen that an individual oscillator of two or three dimension can erroneously show more than one cluster. Averaging out the linearized system retains the appropriate block-diagonal structure. The clustering technique has to be flawless in identifying the blocks. However, the accuracy of Method II and III in capturing the block-diagonal structure immensely depend on the coupling strength and the accuracy of Algorithm~\ref{Eigenvect2}. Once that is done perfectly, the clustering algorithm works well in identifying the block. The accuracy of the methods in identifying decoupled oscillators is very high and results in very small error values. The small error values arise due to numerical computation. With increase in the value of $\epsilon$, the effectiveness of the Statistical Linearization increases. 

For a fixed initial condition, the Jacobian-based linearization techniques involve a deterministic scheme to form the graph adjacency matrix. Thus, the randomness in the technique is induced from the randomness of the clustering technique. The last step of Spectral Clustering, which involves k-means clustering, and NMF, both require generating an initial random seed. Also, as it has been pointed out, if the clustering technique fails, even at a single time instance, then cumulative error becomes huge. 

\textit{What is the computational expense of each of the methods?} Out of the seven proposed techniques, four of them are based on Eigenvalue computation. Method II and III of linearization involve Eigenvalue computation at each step. With the increase in dimensionality, the Eigenvalue computation becomes very expensive. Also, the convergence of numerical methods of Eigenvalue computation for a high dimensional sparse matrix is itself a challenging problem. On the other hand, the Bayesian NMF requires an iterative procedure for finding the best possible solution, which requires much less computational time than the Eigenvalue computation. Linearization method I-III also requires the estimation of the nominal trajectory. The one-point Statistical Linearization proves to be computationally inexpensive compared to the other methods. 

\textit{What are the effects of different factors?}  The ANOVA (Table~\ref{anova}) shows the effect of factors on the performance of the algorithms. The p-values against the effect of $N$, $e$, and methods of Linearization and Clustering show the robustness of the method. Despite showing apparent differences in the numerical values, the contributions of each of these factors to the variability of the performance is shown to be statistically insignificant. Thus, we fail to reject the null Hypothesis, that the error generated in the current methodology is independent of the dimensionality, coupling strength, and the method of linearization and clustering. The p-values against the other main effects seem to quite high. Thus, we reject the null hypothesis for `Type of Problem'. 

\textit{What if the coupling functions are different or the coupling strength is high?} The current study is restricted to diffusive coupling. In the linear system, due to the cyclic diffusive coupling function, one is able to derive suitable properties and get a good understanding of the system. The algebra of the system helps us to infer the pattern of the propagation of the cluster. With that idea, diffusive coupling has been used in nonlinear systems also. The coupling strengths have been reduced so as to obtain proper cluster structure. It has also been noted that with the increase in coupling strength, the value of error increases. The simple Jacobian can be replaced by the State Transition Matrix (STM), and it can be hypothesized, that the effect of coupling can be better captured. STM gives an advanced idea of the future cluster structure with respect to the previous state. Hence, the predictive model can be made better. However, with STM, the computational expense for the Statistical Linearization-based method becomes exceptionally huge. 



Chapter~\ref{chap:scs} relaxes the binary association theory the WCS framework to a fuzzy association framework to detect Strongly Connected Subsystems (SCSs). In this context following research questions have been addressed in the dissertation:

%\textit{How accurate is the assumption of the summation of element-to-element product?}
The accuracy of the element-to-element or Hadamard product lies in the assumption of linear relationship of the estimated association values with the adjacency matrix. Thus, the method accurately estimates the deterministic and as well as uncertain flow in the Shallow Water model because of the linearity in the original model of Equation~\ref{swe_discrete}. For a nonlinear model, the association values are to be periodically updated with availability of measurement, or after a certain period of time. 

In the work reported in Chapter~\ref{chap:building}, a reduced-order thermal model has been developed, using the lumped capacitance RC network model to calculate the load the HVAC system. Using the RC network model in conjunction with the Weakly Connected Subsystems optimal estimation method, we are able to provide a condensed framework for estimating internal loads, solar heat gains, and HVAC supply air temperatures in BEM.  Furthermore, one can use the developed method to calculate the uncertainty of the results for a comprehensive representation of the building systems.
One significant advantage of the presented methodology in Chapter~\ref{chap:building} is the ability to reduce the computation expense of large-scale dynamical systems while maintaining accuracy and providing uncertainty information at each step.  

It is shown that the overlapping clustering method detailed in Chapter~\ref{chap:scs} is proved to be effective for not only state space decomposition but also for clustering a random field. Each cluster of random field is shown to require less number of eigenvalues for approximate KL expansion when compared the whole field. Thus, the clustering not only aids in reduced sample points for the state space but also for the random field.

%As mentioned in Chapter~\ref{chap:wcs}, the search for an ideal clustering technique is never ending. It is to be noted that in the later two chapters, the preferred method of clustering got shifted to Louvain Modularity Optimization than SC or NMF. 

The two application problems used in this dissertation are in two somewhat distinct engineering domain. The applicabilities of the developed methods show the usage of the \textit{linearization and clustering} methods in UQ of most of the physics-based model. As mentioned earlier, the SCS-based UQ method did not show a significant improvement in the performance for the coupled oscillator problem. This result motivated us to use the WCS-based UQ method for the BEM problem. Results are promising and can be interpreted very easily. The WCS-based UQ was earlier applied to PDE problems without much success. Due to the failure of WCS-based UQ approaches to meaningfully solve PDE problems, the results of applied WCS based UQ methods on PDE problems were not reported in this dissertation. The development of the SCS-based UQ methods allows us to tackle the PDE problems. The linear Shallow Water Equation showed satisfactory performance with the use of SCS-based UQ method. The DEM problem is also shown to work very effectively within the SCS-based UQ framework. It is to be noted that the applicability of the SCS-based UQ slightly differ in the 1D linear Shallow Water Equation than the 2D Geophysical Mass Flow equation. 

\section{Future Works}

Analyzing the overall work, some of the future works pertaining to the dissertation are as follows:

\textit{What are the physical interpretation of the cluster structure?} The discussed methods are very instrumental in detecting decoupled oscillators when the coupling strength is 0. However, the current methodology cannot guarantee absolutely zero error for highly coupled systems. Also, the current theoretical limitation compels us to resort to perform repetitive numerical simulations and statistical analysis. Ascertaining whether the identified WCSs according to the current framework has any physical interpretation in terms of the dynamics of the system is a scope of future work. 

\textit{Are the methods of linearization exclusive?} The four methods of linearization are instantaneous operators averaged over a given time window. As pointed out earlier, replacing the Jacobian with the State Transition Matrix can give us a better insight into the cluster structure, and is a scope of future work.  

\textit{How does the Hadamard product ensure the state-space decomposition for a nonlinear system?} Currently the method relies on an intuitive assumption and proof by numerical methods only. The 1D Shallow water equation shows good result for both deterministic and probabilistic cases. A rigorous mathematical proof showing the convergence of the Hadamard product to the actual result for a linear system and its extension to nonlinear system can be a scope for future work. The nonlinearity in the geophysical mass flow problem has restricted us from performing the state-space decomposition. A decomposition could unnecessarily bring complication in tackling the boundary conditions of the subproblems, if not properly defined. 

A repository of the codes summarizing the framework presented in this dissertation is currently in preparation. This repository is to include Statistical Linearization as the preferred method of linearization. The current choice of clustering method is Louvain Modularity Optimization. The repository is also designed to include different sampling and quadrature/cubature-based UQ methods. A Message Parsing Interface (MPI) support is also being designed for this research. Please check the following link for regular updates:
\begin{center}
{\tt \url{https://github.com/arpanisi/UQ_Dynamics}}.
\end{center}


