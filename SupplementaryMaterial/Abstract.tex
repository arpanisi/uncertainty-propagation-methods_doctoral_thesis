The behavior of large nonlinear dynamic systems underlying high-dimensional complex networks is hard to predict. Computational methods become the obvious choice of solution, since the implementation of analytic methods are almost impossible. Even with the development of modern high-computing facilities, aiming for an approximate solution to a high-dimensional physics-based model at times becomes an intractable problem. Uncertainty Propagation (UP) or Uncertainty Quantification (UQ) in such high-dimensional systems by conventional methods requires high computational time and the accuracy obtained in estimating the state variables can also be low. In this context, the primary focus of this dissertation is to explore efficient solutions of UQ problems in solving high dimensional dynamical systems. 

The dissertation outlines two novel methods of solving such a large scale UQ problems. The first work presents a novel computational method focused on performing effective Uncertainty Quantification (UQ) in large networked systems comprising of weakly coupled subsystems (WCSs). They key to the outlined approach is to model complex dynamical systems using graph (network) representation, whose nodes represent the state-space units, and whose links stand for the interactions between them. In this respect, a novel framework is outlined that integrates the concept of time domain and space domain linearization techniques with graph clustering algorithms to identify WCSs in high dimensional complex networks. The outlined technique enables identification of WCSs and thus facilitates effective UQ by parallelizing sampling process and reduce computational time for quadrature-based methods.  This work also highlights on the review and analytic comparison of a couple of clustering techniques (Spectral Clustering (SC) and Non-negative Matrix Factorization (NMF)) that are applicable in the domain of Uncertainty Quantification (UQ) of Large Scale Dynamical Systems. Additionally, a new metric has been developed to compare the performance of the clustering algorithms in the UQ domain. Also, key factors that affect the performance of the algorithms have been identified through the use of comprehensive statistical analysis.

The second key contribution of this dissertation is the development of a methodology for decomposing the state space of high dimensional dynamical systems with strong inter-state coupling. The outlined approach intends to make rigorous UQ of high dimension problem \textit{feasible} by partitioning the overall high dimensional state space problem into multiple lower dimensional state space problems. To enable accelerated and scalable UQ in high dimensional complex physical system models, the second proposed decomposition method leverages an overlapping community detection to detect state variables participating in more than one subsystems (clusters). In contrast to WCS, these subsystems are termed as Strongly Coupled Subsystems (SCS). The final UQ solution is obtained by using the concept of Hadamard product of the state variables in a subsystem (cluster) and their association in the cluster. The developed approach has been tested to detect connected subsystems in coupled dynamical systems. The results analyzing spatio-temporal flow equation are also presented. It is also shown that proposed framework approach is faster and works with less memory requirement to carry out UQ of high dimensional physical system models. 
	
To establish the usefulness of the state-space decomposition technique, the WCS identification-based approach has been applied to estimate thermal load in a large scale office/college building. A building comprises of several well connected thermodynamic zones that are characterized by different performing variables and their conditions of operation. Interpretation of underlying physics in such models is non-trivial. Recent developments of dynamical system models help in estimating thermal loads over an extended period of time in a building energy simulation problem. However, due to simplification of the actual physical process, there might be uncertainties developed in the simulation of such models. Uncertainty Quantification (UQ) in such problems is intractable due to the size of the problem and parameters involved and hence such problems serve as a good benchmark applied problem for the developed methodology

Finally, the developed novel SCS-based method is used to enable UQ of Digital Elevation Models (DEM). With the recent advent of aerial photography, capturing high-resolution terrain information has provided new opportunities to simulate geophysical mass flow on high-resolution Digital Elevation Models (DEM). 
This gives a better understanding of the flow of hazards that have wide range of size. However, performing Uncertainty Quantification (UQ) of such flow on uncertain terrain profile still poses significant computational challenges. 
Even though there exists advanced statistical methods to model the DEM, UQ on the DEM requires generation of huge number of realizations that makes the problem computationally prohibitive. 
This section focuses on the usefulness of the SCS-based method to create a parallel sampling scheme for studying a high-resolution DEMs. 
The realizations sampled in parallel are used to propagate the uncertainty in DEMs using simulations performed by Titan2D to estimate the uncertainty in the pile height.
The accuracy of the SCS-based UQ framework in estimating hazard maps is demonstrated by applying it on the 1991 block-and-ash flows resulting from the 1991 Colima Volcano, Mexico. 

\textbf{Keywords: }Stochastic Dynamical Systems, Approximate Linearization, Graph-Clustering Methods, Weakly Connected Subsystems, Strongly Connected Subsystems, Building Energy Simulation, Geophysical Mass Flow

